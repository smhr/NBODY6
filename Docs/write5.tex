\magnification=1200
\font\big=cmr10 scaled \magstep2
\vglue 1.5cm
\hsize=6truein
\vsize=8.5truein
\centerline {\big NBODY5}
\bigskip
\bigskip
\centerline {REGULARIZED AHMAD--COHEN N-BODY CODE}
\medskip
\centerline {WITH TRIPLE AND BINARY COLLISIONS}
\bigskip
\centerline {Sverre J. Aarseth, Institute of Astronomy, Cambridge}
\bigskip
\bigskip
\centerline {INTRODUCTION}
\bigskip

         The basic methods of solution are described in three papers:
\parindent 40pt
\bigskip

             Ahmad, A. and Cohen, L. 1973, {\it J. Comp. Phys.} 12, 389.

             Aarseth, S. J. 1971, {\it Astrophys. and Space Sci.} 14, 118.

             Aarseth, S. J. 1985, in {\it Multiple Time Scales}, p. 377.
\bigskip
\parindent 12pt

       The first paper describes the direct integration method for the N-body
    problem.  It is based on the idea of separating the total force on a
    particle into two parts due to neighbours and distant particles.  Each part
    is represented by a fourth-order force polynomial with its own
    time-step.  The irregular
       neighbour force is evaluated at frequent intervals and the
    regular distant force contribution is then added by prediction unless the
    next time for total force calculation has been reached.  In the latter case
    the neighbour list is updated and consistent corrections are applied to the
    irregular and regular force differences (where appropriate).  This procedure
 is more efficient than the standard polynomial method for particle
    numbers exceeding about 50.

   The second paper describes the Kustaanheimo-Stiefel (KS) two-body
    regularization method which is used for treating an arbitrary number of
    close hyperbolic encounters and persistent binaries.  The Ahmad--Cohen
 difference scheme has replaced the original fourth-order fitting polynomial
 using Newtonian extrapolation.  Only
 neighbours giving significant perturbations are included in the
    perturbing field.  Conversely, an equivalent c.m. approximation is used when
 calculating the force due to distant regularized components.  A
 complete solution of the two-body
    motion also requires integration of the corresponding centre of mass.  The
    initialization and solution method is the same as for the single particles,
    except that the irregular force is obtained as the mass-weighted sum over
 both components.  A procedure for unperturbed two-body motion is now included
 and Baumgarte stabilization has been implemented for energetic binaries.

    The 1971 paper contains procedures for automatic decision-making
    which enable the two types of solutions to be optimally employed.  The
        switch from direct to regularized solutions is determined by a
  time-step criterion, whereas regularizations are controlled by the relative
   two-body perturbation.  The 1971 paper also explains the organization of
    COMMON variables.  However, the original initialization procedure 
    for regularized solutions has now been superseded by an explicit method.

   The last paper describes minor improvements of the Ahmad--Cohen scheme.  It
 also contains new algorithms for implementing the KS regularization in
 the Ahmad--Cohen scheme.  All regularization procedures are discussed in
 some detail and several modifications have been added.
This paper also describes special sub-programs for three-body and
four-body regularization as well as temporary merging of stable
hierarchical triples (stable quadruples were added later).

  The subsequent sections contain a general description of the present version,
 which is designed for realistic simulations of open star clusters.  Current
 features include the galactic tidal field, supernova mass loss and interstellar
 clouds.  These are optional and the program is also suitable for isolated
 systems.
\bigskip
\bigskip
\centerline{                            GENERAL INFORMATION}
\bigskip

   The program is written in FORTRAN IV and has been developed on the IBM
 370/165 {\it\&} VAX 11/780.  Although programming efficiency has been a
 primary concern, direct N-body calculations are time-consuming.  It is
 therefore desirable to use a high-level compiler.

        The program uses full double precision arithmetic,
    appropriate to machines with less than ten figure accuracy.  With 
 mixed single and double precision it
 is possible to study a 1000-body system with up to 39 neighbours and 10
    simultaneous regularizations on a 500 K byte machine using phase overlay and
    optimizing compiler (program NBODY4).  Program NBODY5 needs an 
    extra 179 K bytes.

   Access to the tape/disc routine MYDUMP is by CALL MYDUMP(I,J).  If I = 0 all
 COMMON blocks are read from logical unit J, and I $>$ 0 is used for saving
 COMMON.  The restart procedure requires that all important variables are stored
 in COMMON.  Several labelled COMMON blocks are used for organizational
 convenience and the whole standard COMMON is added to most routines
 by the INCLUDE 'COMMON5.FOR' statement which is slightly non-standard.
 The standard COMMON block is supplied as a separate file.
 Logical unit 5 is used for input and unit 6 for output,
              whereas units 1, 2 {\it\&} 3 denote tape/disc.   

   A considerable effort has gone into the development of strategies which
 optimize the running time without unduly sacrificing accuracy.  Apart from
 computer word limitations, increased accuracy can be achieved by changing the
 input parameters (described below).  The main exceptions are that routines
 START and FPOLY neglect the relatively time-consuming second and third force
 derivatives for the most distant particles, and that large third-order derivative
 corrections are sometimes omitted in routine INTGRT to avoid spurious reduction
 of regular steps.

   The distinction between neighbours and distant particles is central to the
 present method.  New neighbours are determined at appropriate intervals and
 the correction procedures for neighbour changes are consistent up to the
 highest order in the force polynomials.  For every particle I, an array
 LIST(J,I) holds the locations of neighbours sequentially in J = 2,3,...,NNB+1,
 where NNB = LIST(1,I) is the current neighbour number.  A consistent updating
 of the relevant neighbour lists is carried out at the beginning and end of
 each regularization.  This procedure permits a simple organization of the
 COMMON variables (see the 1985 paper for details).
\bigskip
\bigskip
\centerline {ROUTINES}
\medskip
\hrule
\medskip
\settabs\+\indent&NSTEPNX\quad&\cr
\+&MAIN&      Main calling sequence and reading of restart data. \cr
\+&MYDUMP&    Reads or writes all COMMON blocks (option 1 {\it\&} 2). \cr
\+&START&     Input and polynomial initialization. \cr
\+&DATA&      Generation of initial conditions. \cr
\+&CLOUD&     Initialization of interstellar clouds (option 13). \cr
\+&MAINPT&    Output and error control. \cr
\+&CORE&      Core radius {\it\&} density centre (added October 1987). \cr
\+&INTGRT&    N-body integrator for direct and regularized solutions. \cr
\+&TRANSF&    Initialization of regularized pair. \cr
\+&FINISH&    Termination of regularized pair. \cr
\+&RESOLV&    Transformation of regularized variables to physical variables. \cr
\+&FPOLY&     Polynomial initialization for components of old pair or new c.m. \cr
\+&ESCAPE&    Removal of escaping particles. \cr
\+&MLOSS&     Mass loss from evolving stars (option 19). \cr
\+&TRIPLE&    Sub-program for three-body regularization (option 5). \cr
\+&TETRAD&    Sub-program for four-body regularization (option 5). \cr
\+&MERGE&     Sub-program for hierarchical triples {\it\&} quadruples (option 5). \cr
\medskip
\hrule
\bigskip
\bigskip
\centerline {COMMON VARIABLES}
\medskip
\hrule
\medskip
\settabs\+\indent&NSTEPNX\quad&\cr
 \+&A&     General purpose working space. \cr
\+&ADP&     Working space for regularized solutions. \cr
\+&A1&      Generalized Levi-Civita matrix. \cr
\+&BE&      Total energy.  BE(1)=initial, BE(2)=previous, BE(3)=current. \cr
\+&BETA&    Maximum regularized time-step (superseded June 1988). \cr
\+&BODY&    Mass of an individual particle. \cr
\+&BODYCL&  Current mass of interstellar clouds (initialized to zero). \cr
\+&CLDOT&   Inverse time-scale for cloud mass growth or decay. \cr
\+&CLM&     Effective mass of clouds. \cr
\+&CLMDOT&  Time derivative of cloud masses (used near the boundary). \cr
\+&CMR&     Density centre and scalar value (based on maximum density). \cr
\+&CMRDOT&  Centre of mass velocity components and scalar value. \cr
\+&CMSEP2&  Distance ratio for c.m. approximation (square is stored). \cr
\+&CPU&     Maximum computing time in minutes (used with the timer). \cr
\+&DELTAN&  Regularized time-step parameter (6.28$\ast$DELTAN steps/orbit). \cr
\+&DELTAT&  Constant time interval for output. \cr
\+&DFMAX&   Maximum perturbation for soft regularized binaries if R $>$ R0. \cr
\+&DFMIN&   Relative two-body perturbation for unperturbed motion. \cr
\+&DTAU&    Regularized time-step. \cr
\+&DTLIST&  Interval for updating the time-step list. \cr
\+&DTMIN&   Close encounter time-step for regularization search. \cr
\+&D1&      First divided difference of the irregular force. \cr
\+&D1HDOT&  First divided difference of HDOT. \cr
\+&D1R&     First divided difference of the regular force. \cr
\+&D1U&     First divided difference of the regularized force. \cr
\+&D2&      Second divided difference of the irregular force. \cr
\+&D2HDOT&  Second divided difference of HDOT. \cr
\+&D2R&     Second divided difference of the regular force. \cr
\+&D2U&     Second divided difference of the regularized force. \cr
\+&D3&      Third divided difference of the irregular force. \cr
\+&D3HDOT&  Third divided difference of HDOT. \cr
\+&D3R&     Third divided difference of the regular force. \cr
\+&D3U&     Third divided difference of the regularized force. \cr
\+&ECLOSE&  Binding energy per unit mass for hard binary (positive). \cr
\+&ETAI&    Time-step parameter for irregular force polynomial. \cr
\+&ETAR&    Time-step parameter for the regular force polynomial. \cr
\+&F&       One-half the total force at last step. \cr
\+&FDOT&    One-sixth the total force derivative at last step. \cr
\+&FI&      Irregular force due to the neighbours. \cr
\+&FR&      Regular force due to the distant particles. \cr
\+&FU&      One-half the regularized force. \cr
\+&FUDOT&   One-sixth the regularized force derivative. \cr
\+&F1&      Total force on the current particle. \cr
\+&F1DOT&   First force derivative of the current particle. \cr
\+&F2DOT&   Second force derivative of the current particle. \cr
\+&F3DOT&   Third force derivative of the current particle. \cr
\+&F4DOT&   Fourth force derivative of the current particle. \cr
\+&GAMMA&   Relative two-body perturbation of a regularized pair. \cr
\+&H&       Binding energy per unit mass of a regularized pair. \cr
\+&HDOT&    First regularized derivative of H. \cr
\+&ICOMP&   Global index of the first regularized component. \cr
\+&ILIST&   Current list of neighbours. \cr
\+&IPHASE&  Control variable for the calling sequence (permits overlay). \cr
\+&JCOMP&   Global index of the second regularized component. \cr
\+&JLIST&   Current list of receding neighbours in shell within 1.26$\ast$RS. \cr
\+&KZ&      Options for alternative paths at run time (see table). \cr
\+&LIST&    Neighbour list for perturbed motion (LIST(1,I) = membership). \cr
\+&LISTR&   List of components for recently disrupted pairs. \cr
\+&LISTV&   List of high-velocity particles (neighbour detection). \cr
\+&N&       Current particle number (maximum is 1000). \cr
\+&NAME&    Initial particle name (set $<$ 0 for merged ghosts). \cr
\+&NCL&     Number of clouds (maximum is 5). \cr
\+&NCRIT&   Final particle number (termination criterion). \cr
\+&NDUMP&   Control variable for automatic restart procedure (option 2). \cr
\+&NEWCL&   Number of clouds initialized on the boundary. \cr
\+&NFIX&    Output frequency of binaries or data save (options 3, 6, 18). \cr
\+&NLIST&   Time-step list for determining next body to be integrated. \cr
\+&NNBMAX&  Maximum number of neighbours (two extra for c.m. splitting). \cr
\+&NPAIRS&  Current number of regularized pairs (maximum is 10). \cr
\+&NPRINT&  Output counter (reset to zero when NPRINT = NFIX). \cr
\+&NSTEPN&  Counters for diagnostic output information (see table). \cr
\+&NSTEPS&  Time-step counter for safety dumps (reset each 100000 steps). \cr
\+&NTIMER&  Step counter for calling clock (reset every 1000 steps). \cr
\+&NTOT&    Total number of direct solutions (NTOT = N + NPAIRS). \cr
\+&NZERO&   Initial particle number. \cr
\+&ONE2&    The constant 0.5. \cr
\+&ONE3&    The constant 1.0/3.0. \cr
\+&ONE4&    The constant 0.25. \cr
\+&ONE5&    The constant 0.2. \cr
\+&ONE6&    The constant 1.0/6.0. \cr
\+&ONE12&   The constant 1.0/12.0. \cr
\+&PCL2&    Minimum cloud impact parameter squared (reset every output). \cr
\+&P1&      Momentum of relative two-body motion. \cr
\+&QE&      Relative energy tolerance for error control. \cr
\+&Q1&      Coordinates of relative two-body motion. \cr
\+&R&       Current two-body separation of a regularized pair. \cr
\+&RB2&     Radius of cloud boundary (square is stored). \cr
\+&RCL2&    Half-mass radius of clouds (square is stored). \cr
\+&RMIN&    Distance parameter for regularization search {\it\&} termination. \cr
\+&RSCALE&  Half-mass radius (obtained from effective potential energy). \cr
\+&RTIDE&   Tidal radius (2$\ast$RTIDE used for escaper removal). \cr
\+&RS&      Radius of the standard neighbour sphere. \cr
\+&R0&      Initial two-body separation of a regularized pair. \cr
\+&SIGMA&   Gaussian velocity dispersion of clouds. \cr
\+&SMIN&    Close encounter time-step for neighbour retention. \cr
\+&STEP&    Time-step for the irregular force polynomial. \cr
\+&STEPCL&  Time-step for integration of clouds. \cr
\+&STEPR&   Time-step for regular force polynomial. \cr
\+&TAU&     Regularized time. \cr
\+&TCL&     Time of the last cloud integration (option 13). \cr
\+&TCR&     Standard crossing time (obtained from total mass {\it\&} energy). \cr
\+&TCRIT&   Termination time in scaled units. \cr
\+&TDOT2&   Second regularized derivative of the time. \cr
\+&TDOT3&   Third regularized derivative of the time. \cr
\+&TIDAL&   Tidal parameters (X, Z, {\it\&} Coriolis terms; also RBAR in pc). \cr
\+&TIME&    Physical integration time in scaled units. \cr
\+&TLIST&   Time for next updating of the time-step list. \cr
\+&TNEXT&   Time for next output. \cr
\+&T0&      Time of the last irregular force calculation. \cr
\+&T0R&     Time of last regular force calculation. \cr
\+&T0U&     Time of the last regularized force (regularized units). \cr
\+&T1&      Time of the first backwards irregular force calculation. \cr
\+&T1R&     Time of first backwards regular force calculation. \cr
\+&T1U&     Time of the first backwards regularized force calculation. \cr
\+&T2&      Time of the second backwards irregular force calculation. \cr
\+&T2R&     Time of second backwards regular force calculation. \cr
\+&T2U&     Time of the second backwards regularized force calculation. \cr
\+&T3&      Time of the third backwards irregular force calculation. \cr
\+&T3R&     Time of third backwards regular force calculation. \cr
\+&T3U&     Time of the third backwards regularized force calculation. \cr
\+&U&       Current regularized coordinates of a pair. \cr
\+&UDOT&    Regularized velocities at the beginning of a step. \cr
\+&U0&      Regularized coordinates at the beginning of a step. \cr
\+&V&       Regularized momentum. \cr
\+&VCL&     Initial mean velocity of clouds (constant if SIGMA = 0). \cr
\+&X&       Cartesian coordinates at current time. \cr
\+&XCL&     Cartesian coordinates of clouds at current time. \cr
\+&XDOT&    Velocities at beginning of step {\it\&} current values at output. \cr
\+&XDOTCL&  Velocities of clouds. \cr
\+&X0&      Coordinates at the beginning of a step. \cr
\+&X0DOT&   Velocities at the beginning of a step. \cr
\+&ZMASS&   Current total mass. \cr
\+&ZMBAR&   Mean stellar mass in solar masses. \cr
\medskip
\hrule
\bigskip
\vfill\eject
\centerline {INPUT PARAMETERS}
\medskip
\hrule
\medskip
\settabs\+\indent&NSTEPNX\quad&\cr
 \+&KSTART*&  Values 1,2,3 indicate new case, restart, or modified restart. \cr
 \+&TCOMP&   Maximum computing time in minutes. \cr
 \+&N*&       Total particle number. \cr
 \+&NFIX&    Output frequency of binaries or data save (options 3, 6, 18). \cr
 \+&NCRIT&   Final particle number (alternative termination criterion). \cr
 \+&NRAND&   Random number sequence skip. \cr
 \+&NNBMAX&  Maximum number of neighbours (allow two extra locations). \cr
 \+&ETAI*&    Time-step parameter for irregular force polynomial. \cr
 \+&ETAR&    Time-step parameter for regular force polynomial. \cr
 \+&RS0&     Initial radius of neighbour sphere. \cr
 \+&DELTAT&  Output time interval in units of the crossing time. \cr
 \+&TCRIT&   Termination time in units of the crossing time. \cr
 \+&QE&      Energy tolerance (restart if DE/KE $>$ $5\ast$ QE {\it\&} KZ(2) $>$ 0). \cr
 \+&SMIN&    Close encounter time-step for neighbour retention. \cr
 \+&RBAR&    Mean cluster radius in pc (use RBAR = 0 for isolated cluster). \cr
 \+&ZMBAR&   Mean mass in solar units (use ZMBAR $>$0 for isolated cluster). \cr
 \+&KZ(J)*&   Non-zero options for alternative paths (see separate table). \cr
 \+&DTMIN*&   Time-step parameter for regularization search. \cr
 \+&RMIN&    Distance parameter for regularization search. \cr
 \+&BETA&    Maximum regularized time-step (superseded June 1988). \cr
 \+&DELTAN&  Regularized time-step parameter (6.28$\ast$DELTAN steps/orbit). \cr
 \+&CMSEP2&  Distance ratio for c.m. approximation (square is stored). \cr
 \+&ECLOSE&  Binding energy per unit mass for hard binary (positive). \cr
 \+&DFMIN&   Relative two-body perturbation for unperturbed motion. \cr
 \+&DFMAX&   Maximum perturbation for standard regularization termination. \cr
 \+&ALPHA*&   Power law index for initial mass function (routine DATA). \cr
 \+&BODY1&   Maximum particle mass before scaling. \cr
 \+&BODYN&   Minimum particle mass before scaling. \cr
 \+&Q&       Virial theorem scaling factor (Q = 0.5 for equilibrium). \cr
 \+&NCL*&     Number of interstellar clouds (routine DATA with option 13). \cr
 \+&RB2&     Radius of cloud boundary in pc (square is stored). \cr
 \+&VCL&     Initial mean cloud velocity in km/sec. \cr
 \+&SIGMA&   Gaussian velocity dispersion of clouds in km/sec. \cr
 \+&CLM&     Individual cloud masses in solar masses (maximum 5 clouds). \cr
 \+&RCL2&    Half-mass radii of clouds in pc (square is stored). \cr
\medskip
\hrule
\medskip
\+& &    *) New input line (free format).  \cr
\bigskip
\vfill\eject
\centerline {OPTIONS KZ(J)}
\medskip
\hrule
\medskip
\settabs\+\indent&XXX\quad&\cr
\+&~~1&  COMMON save on unit 1 if TCOMP $>$ CPU or if TIME $>$ TCRIT. \cr
\+&~~2&  COMMON save on unit 2 for restart if DE/KE $>$ 5$\ast$QE. \cr
\+&~~3&  Output of significant non-regularized binaries (frequency NFIX). \cr
\+&~~4&  Skips full predictor loop if neighbour number $>$ KZ(4) {\it\&} KZ(4) $>$ 0. \cr
\+&~~5&  Triple {\it\&} tetrad regularization or mergers (full output if $>$ 1). \cr
\+&~~6&  Essential data written to unit 3 at output time (frequency NFIX). \cr
\+&~~7&  All bodies printed at output time. \cr
\+&~~8&  Diagnostic output for c.m. particles at regular steps. \cr
\+&~~9&  First five bodies printed at output time. \cr
\+&10&  Diagnostic output at beginning of regularization (routine TRANSF). \cr
\+&11&  Diagnostic output at end of regularization (routine FINISH). \cr
\+&12&  Diagnostic output at each regularized step (routine INTGRT). \cr
\+&13&  Interstellar clouds (output if $>$ 1.  Gaussian if $<$ 0 or $>$ 2). \cr
\+&14&  Stabilization of regularized motion for hard binary (H $<$ --ECLOSE). \cr
\+&15&  Neighbour list modifications printed in routine TRANSF. \cr
\+&16&  Neighbour list modifications printed in routine FINISH. \cr
\+&17&  Modification of ETAI {\it\&} ETAR by QE if $>$ 0 and RMIN {\it\&} DTMIN if $>$ 1. \cr
\+&18&  Output of regularized binaries (frequency NFIX). \cr
\+&19&  Mass loss (option = 2 after mass loss to skip next energy check). \cr
\+&20&  Solid-body rotation (if = 1) or Plummer model with f(m) (if = 2). \cr
\medskip
\hrule
\bigskip
\bigskip
\centerline {OUTPUT COUNTERS NSTEPN(J)}
\medskip
\hrule
\medskip
\+&~~1&  Irregular integration steps. \cr
\+&~~2&  Regular integration steps. \cr
\+&~~3&  Retained neighbours using time-step criterion SMIN. \cr
\+&~~4&  Force polynomial corrections. \cr
\+&~~5&  Too many neighbours with standard criterion. \cr
\+&~~6&  No neighbours inside 1.26 times the basic radius. \cr
\+&~~7&  Regularization attempts. \cr
\+&~~8&  Total regularizations. \cr
\+&~~9&  Hyperbolic regularizations. \cr
\+&10&  Too many neighbours when replacing c.m. by individual components. \cr
\+&11&  Regularized integration steps. \cr
\+&12&  Coordinate predictions of all particles (see option 4). \cr
\+&13&  Unperturbed binary orbits. \cr
\+&14&  Large F3DOT corrections not included. \cr
\+&15&  Second component of old regularized pair included as neighbour. \cr
\+&16&  C.m. values for polynomial corrections of regularized component. \cr
\+&17&  Second component of recently disrupted pair included as neighbour. \cr
\+&18&  No neighbours with standard criterion. \cr
\+&19&  Maximum permitted regularized pairs (NPAIRS = 10). \cr
\+&20&  Small step neighbour selected from other neighbour lists. \cr
\+&21&  Search for triple {\it\&} four-body regularization or mergers. \cr
\+&22&  Fast particle included in LISTV. \cr
\+&24&  Mergers of stable triples or quadruples. \cr
\medskip
\hrule
\bigskip
\bigskip
\centerline {THE INTEGRATOR}
\bigskip

   The integrator package consists of the routines START, INTGRT, TRANSF,
  FINISH, RESOLV and FPOLY.  It is suitable for studying a variety of dynamical
 problems and only routines START and INTGRT may require modification.  The
 choice of the basic
 integration parameters and initial conditions for routine START are discussed
 in subsequent sections.

   Program control for output and termination is at the end of each
 time-step loop in routine INTGRT.  The output intervals are determined by the
 time condition TIME $>$ TNEXT.  Control is returned to the integration routine
 unless the end of the calculation has been reached.

 VAX system routines are used for timing a run.  After initialization by
 CALL LIB\$INIT in routine MAIN, the elapsed time is obtained by CALL LIB\$STAT
 near label 430 of routine INTGRT.  The timer is
 called when the COMMON variable NTIMER reaches a fixed number of steps
 (here 1000), after which it is set to zero again.  If the elapsed time exceeds
 the value of CPU, specified at run time and option 1 is non-zero, all COMMON
 blocks are saved on logical unit 1 before stopping.  A bit extra time should be
 allowed for output occurring near the end of a run.  The calculation can be
 continued by reading one data card specifying  KSTART = 2 and TCOMP = computing
 time (in minutes).  If KSTART = 3 routine MAIN reads new values for some
 parameters and one option can also be changed.  In either case, all COMMON
 blocks are read and control is returned to routine INTGRT again.

   Some saving of core storage may be achieved by overlaying different parts of
 the program.  The routines can be arranged into segments as follows:
 (1) MAIN, MYDUMP, RESOLV, CLOUD; (2) START, DATA; (3) MAINPT, ESCAPE; (4)
 INTGRT; (5) TRANSF, FPOLY; (6) FINISH, FPOLY1, MLOSS.  Routine FPOLY1 is
 identical to FPOLY and should be called from
 FINISH by its new name if this procedure is adopted.  Segment 1 is the root
 phase; the others are formed by the linkage editor using appropriate control
 cards.  Alternatively routines INTGRT, TRANSF, FINISH, FPOLY {\it\&} MLOSS may form
 one overlay segment.  Routine MAIN controls
 the calling sequence to the different phases by the COMMON variable
 IPHASE set in INTGRT.  This works equally well if overlay is not used.

 The size of the COMMON blocks depends on the parameters N,
 NPAIRS {\it\&} NNBMAX.  With
          the full double precision version, the total COMMON storage
 requirement is 434$\ast$NMAX + 376$\ast$KMAX + 2$\ast$NMAX$\ast$LMAX + 4562
 bytes, where NMAX, KMAX and LMAX denote the maximum value of
 NTOT = N + NPAIRS, NPAIRS and full size of neighbour list, respectively.
 Note that LIST(1,I) contains
 the number of neighbours and that at least two locations should be free after
  specifying NNBMAX.  The current version permits N = 1000, NPAIRS = 10
  {\it\&} LMAX = 40.  It is simple
 to change the COMMON arrays to any other values of N {\it\&} NPAIRS.  
 The standard COMMON blocks are normally supplied as a separate
 file and appended in most routines by the INCLUDE 'COMMON5.FOR'
 statement. A convenient general
 expression for the maximum size of neighbour arrays is given by 
 LMAX = 10 + N$\ast$$\ast$0.5, using NNBMAX = LMAX -- 3 for the maximum neighbour number.

 Optional stabilization of regularized motion for hard binaries (H $<$ --ECLOSE)
  was introduced in August 1982.  This procedure prevents the accumulation of
 systematic energy errors over many periods.  To save
         time in routine INTGRT, some two-dimensional arrays are written
 explicitly three times inside loops, i.e. X(1,I), X(2,I), X(3,I).  Note that
 on a scalar machine such as VAX the array
 X(K,I) is faster than X(I,K) which was used earlier.
\bigskip
\bigskip
\centerline {MAIN}
\bigskip

 The main routine controls the calling sequence.  First it reads
 two decision variables, i.e. KSTART {\it\&} TCOMP.  KSTART = 1,2,3 denotes new case,
 restart from COMMON dump on tape/disc or restart with modified parameters,
 respectively.  The local variable TCOMP specifies the maximum computing time
 in minutes to be used for the run.  It is subsequently stored in the COMMON
 variable CPU since otherwise an over-write would occur at
 restart time.  Timing
 procedures are usually installation dependent and the VAX system statements
 CALL LIB\$INIT used for initialization and CALL LIB\$STAT giving the elapsed
 time may therefore have to be replaced.  The latter is used near the end
 of routines MAINPT {\it\&} INTGRT.

 Once control passes to routine
 INTGRT a return does not occur until the next output time denoted by variable
 TNEXT, except for initialization or termination of regularizations
 or mergers.  The
 program terminates either in routine MAINPT or in INTGRT, depending
 on the conditions TIME $>$ TCRIT (or N $<$ NCRIT), or TCOMP $>$ CPU.  In the latter case
 all COMMON variables are saved if option 1 is activated and the appropriate
 tape/disc unit has been allocated.

 MAIN also makes an optional call of routine MLOSS (option 19) at every output
 interval.  This is a routine for mass loss and may be omitted for 
 many applications.

   Program control for sub-programs TRIPLE, TETRAD and MERGE is
 included at the end of MAIN in order to facilitate phase overlay
 (IPHASE = 4, 5, 6 respectively, set in routine INTGRT).
 Before starting the three-body or four-body regularization,
 each relevant KS pair is terminated by calling routine FINISH.
 Merged hierarchical configurations are terminated by calling
 routine RESET (IPHASE = 7 set in INTGRT).
\bigskip
\bigskip
\centerline {MYDUMP}
\bigskip

 A tape/disc routine is provided for writing or reading the COMMON
 variables.  The general
 call statement is CALL MYDUMP(I,J).  If I = 0 all COMMON blocks are read
 from logical unit J, whereas I $>$ 0 is used for saving COMMON.  The restart
 procedure requires that all important variables are stored in COMMON.  Several
 labelled COMMON blocks are used for organizational convenience.  The size of
 COMMON blocks must be consistent with the other routines.  Note the use of
 full double precision; however, some   
 integer arrays are INTEGER $\ast2$ to save memory.
 The required size of COMMON blocks (standard words) for changing
 arrays is as follows: 20$\ast$NMAX + 66$\ast$KMAX + 90 (/BLOCK1/);
 88$\ast$NMAX + 28$\ast$KMAX (/BLOCK3/);
 (LMAX + 1)$\ast$NMAX/2 + 825 (/BLOCK4/). Here NMAX, KMAX and LMAX
 denote the maximum value of NTOT = N + NPAIRS, NPAIRS and the
 whole neighbour list, respectively.
\bigskip
\bigskip
\centerline {MLOSS}
\bigskip

 Realistic simulations of open star clusters must include a procedure for mass
 loss from evolving stars.  This routine is called from MAIN every output time if option 19 is
 non-zero.  A fitting formula is used to calculate the main sequence time-scale
 for the maximum particle mass.  A length scale of 1 pc is assumed for the
 conversion to physical time if tidal forces are not included.  Two cases of
 instantaneous mass loss are considered if the calculated time-scale exceeds the
 current time.  Above 6 solar masses the remnant is assumed to be a neutron star
 of 1.5 solar masses, otherwise a white dwarf in the range 0.38 -- 1.28 solar masses is specified.

 The case of a regularized component undergoing mass loss is included, with an
 unperturbed binary resolved at a random phase angle.  A
 consistent correction
 of the total energy is calculated but option 19 is also increased to 2 to skip
 the next energy check (when it is reduced to 1 again).  If a regularized
 component suffers mass loss the pair is terminated by calling routines FINISH
 {\it\&} FPOLY.  Remnant neutron stars are given a high velocity to ensure 
 rapid escape due to recoil.
\bigskip
\bigskip
\centerline {MAINPT}
\bigskip

 This routine deals with output and error control.  The current
 coordinates and velocities are first predicted to highest order.  The next loop
 which has N$\ast$(N -- 1)/2 terms evaluates the potential energy.  The energy
 of regularized pairs is included directly for increased accuracy, avoiding
 singularities.  The provisional density centre (from maximum
 neighbour density) is set in CMR(1,2,3).  An improved value is 
 obtained in routine CORE which also gives core radius RC {\it\&} 
 membership NC. Other quantities calculated
 are the average neighbour number (denoted by $<$NNB$>$),
 modified kinetic energy, centre of mass
 coordinates, and Z-component of angular momentum.  If non-zero,
 option 4 is updated to KZ(4) = $<$NNB$>$.  

   The main output lines are mostly self-explanatory. Here CM gives
 the current c.m. values of coordinates and velocities, DE is the
 relative energy error (evaluated with respect to the modified
 kinetic energy), E is total energy, TCR denotes the time in units
 of the current crossing time, $<$R$>$ is the approximate half-mass
 radius, CMAX is maximum density contrast, RDENS is location of
 the density centre, AZ is Z-component of angular momentum, EB/E is
 relative energy of regularized binaries, and T6 is time in million
 years (nominal value for isolated system). All the counters NSTEPN
 are also printed together with the number of regularized perturbers,
 current values of RMIN and DTMIN as well as $<$C$>$ which represents the effective
 neighbour number (different from $<$NNB$>$) and relative cost of
 irregular and regular integration. Moreover, one line is printed
 for each particle if option 7 is activated.  Alternatively the first five
 particles are printed if option 9 is non-zero.  For each particle I is printed:
 I, NAME, BODY, STEP, STEPR, EI, RI, LIST(1,I), XYZ-coordinates (w.r. to
 density centre), XDOT(K,I), irregular force, regular force, {\it\&} RS.  Here EI is the
 Jacobian energy per unit mass and RI is the central distance.  
 Option 3 permits a binary search (frequency NFIX).
  Each line gives: index of components,
 mass of components, binding energy per unit mass, semi-major axis,
 mean motion, separation, central distance, eccentricity, and 
 neighbour number of first component.
 Likewise, option 18 produces a similar line for any regularized 
 binaries, with c.m. neighbour number and relative perturbation  
 added at the end.

 A facility for automatic error control is included (option 2) and works as
 follows.  A restart from the last output time is made with reduced time-step
 parameters if the relative energy error exceeds 5$\ast$QE, where QE is
 specified at input.  If option 17 is also activated and the relative error is
 in the range (QE, 5$\ast$QE) the time-step parameters are merely reduced by an
 appropriate amount, whereas ETAI and ETAR are increased slightly if the error
 is $<$ QE/5.  The program terminates after one unsuccessful restart
 using the COMMON variable NDUMP.  As a safety precaution the program also
 terminates if the error exceeds 5$\ast$QE and option 2 is zero.

     Options 1 and 2 may be used independently if desired.  However the most
 flexible procedure is to have both options non-zero.  In this case the
 calculation can be halted at any time.  Alternatively option 2 may be used
 for error control and termination at output time by moving the test TCOMP $>$
 CPU from INTGRT to MAINPT.  If there are no
 numerical problems it is sufficient to use option 1.

 The velocities XDOT which are over-written must be restored before the
 COMMON save to enable a correct restart.  
 The data on unit 1 {\it\&} 2 is
 temporary, being over-written each time.  It is most convenient to accumulate
 the basic data such as TIME, BODY(I), X(K,I), XDOT(K,I) on a separate tape/disc
 unit.  This is done just before setting XDOT(K,I) = X0DOT(K,I), using option
 6 (frequency NFIX).  With disc instead of tape write NTOT first 
 as a separate record to enable reading.
\bigskip
\bigskip
\centerline {CORE}
\bigskip

   The determination of core radius and density centre is based on
the paper by Casertano and Hut ({\it Astrophys. J.} 298, 80), but
using an rms weighting for the density radius. This procedure
gives a well defined core radius for small samples.
All members inside the half-mass radius are first selected.
Individual densities are then assigned according to the sixth 
nearest neighbour, obtained by sorting square distances.
The density centre is now formed by a density-weighted summation
over the selected coordinates, whereas the core radius is
evaluated from an rms expression involving the sum of central
distances and densities squared.
The current density centre is set in the COMMON variable CMR(K)
which therefore no longer coincides with the particle of largest
neighbour density.

   The following quantities are available upon return to routine
MAINPT: core radius (RC), density-weighted average density (RHOD),
maximum density (RHOM), and particle number inside the core radius (NC).
Both RHOD and RHOM are scaled by the average density determined
from the half-mass radius.
Note that the present algorithm treats all particles equally; the
core mass may therefore be underestimated for unequal masses.
 However, it is not obvious how to generalize the present procedure
 to the case of a mass spectrum.
\bigskip
\bigskip
\centerline {ESCAPE}
\bigskip

 This routine is called from MAINPT.  It removes particles outside twice the
 tidal radius, calculated with respect to the current density centre.
 A simple escape velocity test is added for isolated
 systems.  Corrections of the total energy maintains energy conservation.
 To remove a particle, all
 relevant COMMON variables are updated (including NLIST, LISTR {\it\&} LISTV).  
 A separate procedure removes any
 regularized pairs (i.e. first the c.m., then the components).  
 Merged ghosts (single particle or binary c.m.) are normally  
 retained (skip on RI2 $>$ 1.0D+10), but are removed consistently 
 in case the associated binary is also escaping. For isolated
 cluster calculations RTIDE = $10\ast$RSCALE (set by routine MAINPT), otherwise 
 an appropriate value is calculated from the tidal parameters and 
 updated according to the new total mass.

 The main line of output contains: current particle number, total mass, total
 energy, c.m. scalar value, Y-component of c.m., central distance of last
 escaper, irregular time-step, radius of neighbour sphere, mean mass of
 remaining particles, number of particles inside one tidal radius, and names of
 escapers.  The escape angles with respect to the X-axis and XY-plane are also
 printed if the tidal field is included.
 A separate line is printed for each escaping binary, giving 
 relevant information in obvious notation (note that a ghost
 binary is printed with NAME = 0 and EB = 0).
\bigskip
\bigskip
\centerline {START}
\bigskip

 This routine reads the basic parameters and performs the initialization for
 the integrator.  Relevant counters and variables are first set to zero and the
 fractional constants ONE2, etc are specified.  The reason for introducing these
  constants is to retain flexibility in the choice of precision by means of a
 simple declaration and also to save time on divisions (only ONE3, ONE6 {\it\&} ONE12
 are implemented).

 Six data cards are needed for a standard run.  All input is read
 in free format as READ (5,$\ast$) for convenience but this can readily be
 changed.  The relevant variables are
 listed in the input table and the choice of integration parameters is
 discussed near the end of this write-up.  Initial values for the variables
 BODY(I), X(K,I), XDOT(K,I), where I = 1,2,...,N {\it\&} K = 1,2,3, must be specified
 by routine DATA.  The force polynomials are obtained by the method of
 explicit differencing which only requires the masses, coordinates and
 velocities.  See the 1985 book article for details.

 There are two main loops, both
 requiring N$\ast$$\ast$2 operations.  The force and first derivative due to neighbours
 and distant particles are evaluated separately in one loop.  The choice of
 neighbour sphere radius RS for each particle depends on the density as well as
 the maximum permitted neighbour number (NNBMAX).  If the system is not
 homogeneous it is desirable to modify the individual neighbour
 radius (local variable RS2) at the beginning of the loop, otherwise it will be
 set to the input value RS0.  There is an automatic
 reduction/increase of the neighbour sphere if the membership is not in
 the acceptable range (1, NNBMAX -- 1).

 A second main loop obtains the second and third force derivatives for the
 irregular and regular force components.  This procedure requires the total
 force and first derivative (F {\it\&} FDOT) which are also used by the integrator for
 coordinate predictions.  To save time the second and third derivatives for
 the regular component are only evaluated inside a distance 5$\ast$RS since more 
 distant particles do not contribute significantly, i.e. F2DOT scales as
 1/R$\ast$$\ast$5.

 Contributions from the tidal parameters are added to the forces and
 derivatives if appropriate.  The corresponding equations of motion are
 integrated in rotating coordinates (see {\it Bull. Astron.} 3, 47, 1967 or the
 1985 article).  The
 Coriolis terms form part of the irregular
 force field, whereas the linearized tidal forces (X {\it\&} Z) are included in the
 regular field.  Consistent derivatives are formed by the method of explicit
 differencing.  Note that the effect of clouds need not be included since the
 masses are zero initially.

  Individual integration steps for the irregular and regular force polynomials
 are determined from a relative criterion of the type STEP =
 (ETA$\ast$(F$\ast$F2DOT + FDOT$\ast$$\ast$2)/(FDOT$\ast$F3DOT + F2DOT$\ast$$\ast$2))$\ast$$\ast$0.5,
 where the parameter ETA controls the rate of convergence.  The time-step loop
 initializes the times used for the backwards difference scheme (four variables
 for each type).  Particle names are initialized sequentially by
 NAME(I) = I.  This loop also
 sets X0(K,I) = X(K,I) and X0DOT(K,I) = XDOT(K,I).  These variables denote the
 values at the beginning of a step and are only updated at the end of each
 individual time-step, whereas the variables X(K,I) which represent current
 coordinates are over-written in the predictor.  Note that the variables
 XDOT(K,I) are also used to hold current velocities in the output routine (only
 temporarily).  Finally one-half the total force and one-sixth the total force
 derivative is set in F(K,I) and FDOT(K,I), respectively.  This permits a fast
 coordinate prediction to be made.

 Members of the time-step list are determined in the final loop.  The starting
 value of DTLIST is taken as one-quarter of the average time-step, 
 subject to a membership in the permitted range.  The array
 NLIST contains the index of all particles to be treated during the interval
 (0, DTLIST), with NLIST(1) holding the membership number.  If the initial
 irregular time-steps are equal NLIST should be of size N + 1 (note the
 explicit test at the end and also near beginning of routine INTGRT).  In any
 case NLIST should comfortably exceed N$\ast$$\ast$0.5 to include unfavourable bunching.
\bigskip
\bigskip
\centerline {DATA}
\bigskip

 This routine is called from START and specifies initial conditions for
 BODY(I), X(K,I), XDOT(K,I), where I = 1,2,....,N {\it\&} K = 1,2,3.  The present
 version generates a homogeneous spherical system with isotropic velocities,
 using the VAX random number routine RAN(KKK).  Alternatively, a Plummer 
 model is generated if option 20 is 2 (= 1 for solid-body
 rotation).  Individual masses are selected from a power
 law with specified mass range (ALPHA = 2.35 for Salpeter function and
 ALPHA = 1.0 for equal masses).  The coordinates
 and velocities are then expressed in the c.m. rest frame.  The program works
 for any consistent set of units.  The adopted choice is total mass = N
       and total energy = --N$\ast$$\ast$2/4, giving a 
 mean mass and equilibrium length scale of unity, and crossing
 time TCR = 2$\ast$(2/N)$\ast\ast$0.5.  This entails a simple
 scaling of X {\it\&} XDOT after calculation of the total energy.

 The last part sets the scaled parameters for interstellar clouds if option 13
 is non-zero.  The required input parameters are in physical units, as defined
 in the input table.  Initial conditions for each cloud are obtained by calling
 routine CLOUD.
\bigskip
\bigskip
\centerline {CLOUD}
\bigskip

 Each call generates new coordinates and velocities for a cloud moving inwards
 from the boundary.  The positions are randomized on the boundary sphere and the
 velocities are isotropic.  If option 13 is 1 or 2 the space velocities are
 constant (VCL), otherwise a Gaussian distribution is adopted (dispersion
 SIGMA).  The velocities are converted to rotating coordinates for non-zero
 tidal parameters.  The cloud coordinates are expressed with respect to the
 density centre which may be significantly displaced from the coordinate
 centre.  The cloud mass is set to
 zero on the boundary in order to minimize the force discontinuity and is
 increased on a short time-scale.
\bigskip
\bigskip
\centerline {INTGRT}
\bigskip

 This large routine performs the step by step integration of a point-mass
 N-body system.  The present version permits up to 10 regularized pairs and 1000
 single particles.  Two integration methods are
 used, i.e. KS regularization for close two-body encounters and the
 Ahmad--Cohen neighbour scheme for single particles and c.m. bodies.  Divided
 differences have now replaced the polynomial differences described
 in the 1971 paper.

 The loop starts by selecting from the time-step list the next body to be
 treated (denoted by index I throughout).  Note that the current value 
 of the time is defined by TIME = T0(I) +
 STEP(I).  The time-step list is updated if TIME $>$ TLIST and the corresponding
 interval DTLIST is stabilized on a membership in the range
 (0.5$\ast$NNBMAX, NNBMAX) which is nearly optimal on scalar machines.

 The integration of interstellar clouds is included if option 13 is
 non-zero.  This is done in rotating coordinates and tidal force terms are added
 as appropriate.  The equations of motion include the time-averaged net
 inward force field added as a
 repulsive contribution.  A cloud mass is initialized to zero on the boundary by
 routine CLOUD.  It is then increased over the next 0.05 mean cloud
 crossing time until the full value is
 reached.  Likewise, the mass is decreased at the same rate when crossing the
 boundary outwards and a new cloud is generated in its place when this mass
 reaches zero.  The cloud forces are added to the regular field at each
 total force calculation.

 The case I $<$ 2$\ast$NPAIRS + 1 denotes a regularized
 pair.  The pair index is obtained from IPAIR = 0.501$\ast$(I + 1)
 whereupon I is set to the corresponding c.m. value I = N +
 IPAIR.  Coordinates for the perturbers {\it\&} c.m. are first predicted to
 order FDOT, then to F3DOT if GAMMA $>$ 0.001.  The
 regularized coordinates, velocities {\it\&} binding
 energy are predicted to highest order and current physical coordinates
 obtained by a KS transformation.  If option 14 is non-zero, stabilization of
 regularized motion is included (H $<$ --ECLOSE {\it\&} GAMMA $<$ DFMAX).  The
 perturbation is calculated at label 700,
 with the components of neighbour pairs not satisfying the c.m. approximation
 resolved at label 500.  The new acceleration and higher differences are now
  formed,
 after which the fourth-order semi-iteration is included as a corrector 
 for U, UDOT {\it\&} H.    

   There are two main termination criteria, i.e. GAMMA
 $>$ 0.5 for hard binaries (if the new two-body force is dominant) and GAMMA $>$
 DFMAX together with R $>$ R0 for soft binaries.  The distance condition R $>$ RMIN
 is used to terminate hyperbolic motion (a spurious perturber
 may be assigned in TRANSF).  The regularized time-step is set to
 DTAU = 1.0/(DELTAN$\ast$SQRT(2.0$\ast$$\vert$H$\vert$)) and reduced by the
 perturbation factor 1.0/(1.0 + 1000.0$\ast$GAMMA)$\ast$$\ast$0.33.  A maximum time-step
 $(R_0/M)^{1/2}$/DELTAN (modified by $(R_0/R)^{1/2}$ for $R > R_0$)
 is used for soft binaries and weak
 hyperbolic encounters.  The variable STEP holds the physical value of
 DTAU (used by time-step list).

 If option 12 is non-zero the following diagnostic is printed at 
 regularized steps: IPAIR, TIME, H, R, DTAU, GAMMA, STEP(2$\ast$IPAIR--1),
  LIST(1,2$\ast$IPAIR--1).

 The list of perturbing neighbours for
 binaries is updated at successive apocentre passages.  Neighbours are
 selected from the c.m. list if their relative perturbation $>$ DFMIN by the
 tidal approximation.  Unperturbed two-body motion is adopted during the next
 period if there are no such particles.  Subsequent checks are made at intervals
 which may increase (c.f. label 60).  New polynomials are obtained when
 necessary (label 70).  Note the convention LIST(1,2$\ast$IPAIR--1) = 0 for
 unperturbed motion.  In the case of termination the pair index IPAIR
 is stored in COMMON variable ICOMP to permit routine FINISH being part of a
 phase overlay program.  Termination leads to return followed by a call to
 FINISH (since IPHASE = 2), otherwise control passes to the end of the routine.

   If option 5 is non-zero and the c.m. step is $<$ DTMIN at the
apocentre, a search is made for suitable multiple regularizations
or temporary mergers of hierarchical configurations. A critical 
triple or quadruple encounter is accepted for special treatment by
the AZ or Heggie method as an isolated system for small impact
parameter and external perturbation. The calculations are carried
out by sub-programs TRIPLE or TETRAD called from routine MAIN 
(IPHASE = 4 or 5), whereupon suitable restarts are made. Likewise, stable
hierarchies (triples or quadruples) are merged temporarily on 
return to MAIN (IPHASE = 6).
The latter case is distinguished by a negative c.m. name for which
a modified termination criterion 2$\ast$SEMI -- R(IPAIR) $>$
R0(IPAIR) is used, with R0(IPAIR) including a stability factor of 5.

 The treatment of single particles (2$\ast$NPAIRS $<$ I $<$ N + 1) and
 c.m. particles (I $>$
 N) is nearly identical.  However, there are no regularization searches for the
 latter and the acceleration is also obtained by a different expression.  The
 direct integration cycle starts with coordinate predictions to order FDOT for
 either the neighbours (order F2DOT if STEP(I) $<$ SMIN) or all particles,
 depending on whether the regular force component can be extrapolated or needs
 to be evaluated again.  The local variable IR is used as an indicator for the
 relevant case (i.e. IR = 0 or IR = 1) which is determined by comparing
 TIME + STEP(I) with T0R(I)
 + STEPR(I).  Option 4 is used to save time by restricting the full predictor
 loop to particles with small neighbour numbers (below $<$NNB$>$) since other
 particles are predicted more frequently; i.e. there is a multiplicity factor of
 NNB in the predictor.

 A regularization search for the dominant neighbour (denoted by JCOMP) takes
 place at label 100 if STEP(I) $<$ DTMIN {\it\&} I $<$ N + 1.  Only bodies with STEP(J)
   $<$ 2$\ast$DTMIN are searched but first there is a test for possible dominant c.m.
 particles.  If accepted for regularization both I {\it\&} JCOMP are integrated to
 order F3DOT.  The components are stored in COMMON variables ICOMP {\it\&} JCOMP to
 permit phase
 overlay.  Note the convention ICOMP $<$ JCOMP assumed by routine TRANSF which is
 called after control is returned to MAIN with IPHASE = 1.

 The general case of no regularization continues by evaluating the irregular
 force using a separate part at label 600 for c.m. particles with perturbed
 relative motion.  Note that the components of other neighbour pairs are only
 resolved if the c.m. approximation does not 
 hold.  Coriolis terms are added to the irregular force if the tidal
 parameters are non-zero.  After updating the relevant times
 T0, T1, T2, T3, new divided differences are set for the
 irregular force component.  The fourth difference formed from the new
 and old third difference is used as a corrector.  At
    this stage there are two paths depending on the control indicator.  If
 IR = 0 the current regular force and first derivative are extrapolated from the
 third-order polynomial and added to the irregular force and extrapolated
 first derivative to form the total force and first derivative.  Control
  then passes to the end of the cycle where a new irregular integration
 step is obtained, whereupon another cycle is started.  The alternative path
 described below requires a determination of the total force as well as a
 new neighbour search and associated correction procedure.

 The total force loop evaluates the irregular and regular components
    separately and sets the new neighbours (NNB) in array ILIST.  If the c.m.
 approximation does not hold, the components of pairs inside the neighbour
 sphere (and sometimes also outside) are resolved by a KS transformation after
 prediction of U.  Likewise, if I $>$ N the irregular force is obtained
 by algebraic summation over the components unless the c.m.  
 approximation applies.  Approaching
 particles between RS and 1.26$\ast$RS are also included as neighbours if the
 impact parameter $<$ RS.  Other particles in the outer sphere are
 set in JLIST for selection if ILIST is empty, hence a new
 search is only made if JLIST is empty.  

  If the tidal
 parameters are non-zero the linearized force contributions (X {\it\&} Z) are added to
 the regular force and two-body perturbation.  The Coriolis terms are only
 included in the irregular force (i.e. H not affected).

 If NNB $>$ NNBMAX -- 1 the neighbour sphere radius is reduced and some members are
 removed while their force contributions are updated consistently.  The
 condition NNB $<$ NNBMAX ensures that at least one neighbour can be retained or
 added at a later
 stage.  Note that most neighbours are selected by a distance criterion but
 a more general expression may be needed for very large mass ratios.

  The adopted strategy for choosing the neighbour sphere radius uses
 the square root of the number density contrast 
 RHOS = 2$\ast$(NNB/N)$\ast$(RSCALE/RS(I))$\ast$$\ast$3.  The half-mass radius 
 RSCALE is calculated in routine MAINPT from the effective potential
 energy.  The present procedure aims
 at six neighbours for a density contrast of unity and NNBMAX = 37.  For a
 binary c.m. RS is increased to include perturbers of 
 maximum mass.  The volume change is restricted to 25 \% either way and additionally
 if STEPR(I) $< 0.01\ast$TCR.  The neighbour
 sphere is increased by the smallest radial velocity factor
 RSDOT$\ast$STEPR(I) if NNB $<$ 4 and all neighbours are moving into the outer shell.

 There are several procedures for selecting additional neighbours.  The first
 section near label 305 searches close encounter neighbours (STEP(J) $<$ SMIN)
 for companions not already included.  The second part near label 320 looks for
 the other component of a recently terminated pair with separation $<$
 2$\ast$RMIN.  The third section near label 330 searches for the other
 component of an exchanged pair stored in LISTR.  This procedure is also
 useful when eccentric binaries are regularized and terminated
 frequently.  These three procedures are independent, except that 
 a duplicated search is
 avoided.  Any high-velocity particles (velocity $> 8\ast \ast 0.5$ times rms value) are   
 included in array LISTV (updated consistently in TRANSF {\it\&} ESCAPE)
 and added as neighbours if the impact parameter $<$ RS.

  An efficient search procedure is used to find the loss and gain of
 neighbours at the same time, which is faster than that employed
 previously.  If the step is small
 (i.e. STEP(J) $<$ SMIN) a candidate for removal may be retained to avoid large
 derivative corrections, in which case the relevant force corrections are made.

  The next part obtains new regular force differences.  The procedure is almost
 identical to that for the irregular field.  The
 differences are first evaluated for a constant (i.e. old) neighbour
 field.  This is
 achieved by subtracting the change of neighbour force due to the old
 and new members.  The current total force
 and first derivative are then set and the fourth-order corrector
 is included.  Derivative corrections due to the loss and gain of
 neighbours are now accumulated.  These terms are calculated in a
 separate dual purpose part at the end of the routine.  Note that the
 loss of neighbours must be known before evaluating the regular differences because
 of the retention procedure.  If there has been a change of neighbours the
 membership list is updated and consistent corrections performed to the force
 polynomials.  New regular and irregular time-steps are obtained 
 from a composite expression of the type STEP =
 (ETA$\ast$(F$\ast$F2DOT + FDOT$\ast$$\ast$2)/(FDOT$\ast$F3DOT +
 F2DOT$\ast$$\ast$2))$\ast$$\ast$0.5 involving all the derivatives.  
 If desired, a new integration cycle may now be entered at label 1.

  Control passes temporarily to the main routine at every output time, followed
 by a call to MAINPT before returning.  The
 routine ends with a timing check after a specified number (1000) of integration
 steps (both types counted equally).  Provided that option 1 is non-zero, the
 COMMON blocks are saved on unit 1 if the calculation does not finish in
 the specified time (i.e. TCOMP $>$ CPU) and also every 100000 steps.  The
 VAX timer CALL LIB\$STAT is used for this purpose.
\bigskip
\bigskip
\centerline {TRANSF}
\bigskip

 This routine performs the initialization of a new regularized
 pair.  Global variables for the two components ICOMP {\it\&}
 JCOMP are moved up into locations
 2$\ast$NPAIRS + 1 {\it\&} 2$\ast$NPAIRS + 2 by exchange with the latter, 
 whereupon the current value of NPAIRS is increased by one.  Note the convention
 ICOMP $<$ JCOMP imposed in routine INTGRT.  Next follows updating of the
 arrays LIST, LISTR {\it\&} LISTV to be consistent with the
 current particle locations.  These
 procedures are completely general and even the neighbour list 
 modifications are fairly fast.  Note that two regularized components 
 are replaced by the new c.m. in all neighbour lists (one component
 only in any perturber list since no derivative corrections involved).

 Initial conditions for the corresponding centre of mass are added
 at location NTOT = N + NPAIRS, with NAME(NTOT) = NZERO + NAME(ICOMP), and the
 regularized KS variables are introduced at location NPAIRS.  The c.m. force
 polynomial is then obtained by calling routine FPOLY.  Note that the neighbour
 list of ICOMP (without JCOMP) is used for the c.m. initialization which must
 therefore occur after the list updating procedure.

 Regularized KS variables are introduced at location NPAIRS.  Perturbing
 neighbours for the relative motion are selected from the c.m. list.  The
 perturber criterion includes any particle which produces an approximate
 perturbation $>$ DFMIN.  Polynomials for the regularized solution are
 initialized by a
 new method of explicit differentiation which avoids the previous
 cumbersome boot-strapping procedure.  This procedure includes
 the first derivative of the Levi-Civita perturbation term (L$\ast$F) and
 is also satisfactory for large perturbations.  Errors are minimized by taking 
 an initial step of half the standard value
 DTAU = 1.0/(DELTAN$\ast$SQRT(2.0$\ast$$\vert$H$\vert$))
  reduced by the perturbation factor 1.0/(1.0 + 1000.0$\ast$GAMMA)$\ast$$\ast$0.33.  A
 maximum step $(R_0/M)^{1/2}$/DELTAN (modified by
 $(R_0/R)^{1/2}$ if $R > R_0$)
 is used for soft binaries {\it \&} weak hyperbolic 
 encounters.  STEP(ICOMP) holds the corresponding value in physical
 units.  STEP(JCOMP) is set large to avoid JCOMP being included in the
 time-step list.

If option 10 is non-zero the following line is printed: TIME,
 NAME(ICOMP), NAME(JCOMP), DTAU, R, RI, H, IPAIR, GAMMA, STEPI,
 LIST(1,ICOMP), where RI is the central distance and STEPI is the time-step of
 ICOMP.  The termination parameter DFMAX is
 decreased or increased (up to a maximum) if the number of
 pairs is near the limit (modify if NPAIRS $>$ 10).  Finally, the prediction  
 variables X0DOT, F, FDOT, D2, D3, D2R, D3R are set to zero for
 each new component.  This avoids having a skip test on single
 regularized components in the coordinate prediction of routine INTGRT, which now
 gives X = X0 in these rare cases (updated by X0 = X in FINISH).
\bigskip
\bigskip
\centerline {FINISH}
\bigskip

 This routine terminates the regularization of a pair and introduces direct
 integration variables for each component.  The pair index IPAIR is set in
 COMMON variable ICOMP by routine INTGRT to permit phase overlay.  If option 11
 is non-zero the following diagnostic is printed: TIME, BODY(2$\ast$IPAIR--1),
 BODY(2$\ast$IPAIR), DTAU, R, RI, H, IPAIR, GAMMA, STEP(2$\ast$IPAIR--1), and
 LIST(1,2$\ast$IPAIR--1), where RI is the central distance.  Current
 coordinates and velocities for each component are set in routine
 RESOLV.  Tables for subsequent pairs are moved up by one and
 similarly for the corresponding c.m.  The two components are set in
 locations 2$\ast$NPAIRS -- 1 {\it\&} 2$\ast$NPAIRS.

 The second part adjusts the neighbour lists to the new particle sequence and
 updates the array LISTR.  One problem arises if more than two c.m.
 particles are replaced by the components during one regular step and
 NNB = NNBMAX at the beginning.  If NNBMAX = LMAX -- 3 (where
 LMAX is size of LIST), the present version permits one extra
 neighbour to be added after the standard NNB $<$ NNBMAX test, followed
 by two regularization terminations of c.m. neighbours (i.e. addition
 of three extra neighbours).  Any further additions will be suppressed
 but this has not occurred so far (check counter 
 NSTEPN(10)).  Polynomials for the components are initialized by calling
 routine FPOLY.  An identical routine FPOLY1 should be compiled 
 and called if maximum use of phase overlay is desired.
\bigskip
\bigskip
\centerline {RESOLV}
\bigskip

 This is a dual-purpose routine for obtaining the current global coordinates
 and velocities of a specified pair from the regularized variables.  A calling
 argument KDUM = 1 is used when U {\it\&} V (the latter converted from UDOT) must be
 predicted before the transformation to physical variables.  This is required
 at output times and when neighbour pairs are resolved for polynomial
 initialization.  Note that the components of unperturbed binaries (defined by
 LIST(1,2$\ast$IPAIR--1) = 0) are resolved at the apocentre.  The current binding
 energy of IPAIR set in ADP(11) is only required by routines MAINPT and MERGE.  If 
 KDUM = 2 the KS transformation is performed with the
 current values of U {\it\&} V.  The latter path is used for a pair being terminated,
 i.e. called from routine FINISH.  In this case the corresponding c.m. velocity
 is integrated to order F3DOT before the component velocities are set in XDOT
 {\it\&} X0DOT.
\bigskip
\bigskip
\centerline {FPOLY}
\bigskip

 This routine obtains regular and irregular force polynomials for individual
 components after a regularization termination (KCASE = 1) and for the c.m. at
 the start of a regularization (KCASE = 2).  Each component is treated
 separately in either case, the essential difference is that the dominant term
 in F, FDOT, etc is omitted for a calling argument KCASE = 2.  The neighbour
 lists are copied
 from the old c.m. (KCASE = 1) or the first component (KCASE = 2).  The
 respective components are added to the individual neighbour lists which are
 needed in the force polynomial loops.  Current velocities of all neighbours
 (including regularized components) are then set.  Any neighbour pairs with
 perturbed relative motion are resolved in
 the F {\it\&} FDOT loop but not in the F2DOT {\it\&} F3DOT loop which employs the c.m.
 approximation (F {\it\&} FDOT not known for regularized components).  If
 KCASE = 2 the F2DOT {\it\&} F3DOT loop now employs a fast summation over the
 c.m. rather than summation over both components.

 The effect of non-zero galactic tidal parameters is included consistently in
 all polynomial derivatives.  Direct contributions from interstellar clouds are
 also added to the regular force and first difference if required (option 13).

 If KCASE = 2 the c.m. variables for F {\it\&} FDOT are obtained by algebraic 
 summation over the
 components.  Time-steps and force differences for the new c.m. are obtained
 by the expressions used for single particles (as in routine START).  To
 compensate for the single c.m. summation the irregular step is reduced
 by a factor of 2.

 The time-step list is modified to be consistent with the new particle
 sequence.  This saves forming a new list each time (by setting TLIST
 = TIME).  A call from routine FINISH is made with 2$\ast$IPAIR -- 1 {\it\&} N + IPAIR
 in JLIST(1{\it\&}2), whereas routine TRANSF sets the old ICOMP {\it\&} JCOMP in
 JLIST(1{\it\&}2).  These components are first removed from NLIST (if present).  Then
 follows renaming of the first components {\it\&} c.m. particles of more recent
 pairs (KCASE = 1) or renaming of exchanged particles (KCASE = 2).  Finally a
 check is made to add the individual components or new c.m.  The new regularized
 time-step is checked afterwards in routine TRANSF.

 The neighbour velocities are restored to their former value at the end.  It
 seems justified to use old velocities (i.e. XDOT = X0DOT) outside the
 neighbour sphere since the error only affects force derivatives.  Likewise, the
 omission of particles outside 5$\ast$RS from F2DOT {\it\&} F3DOT
 should only have a small effect.

  Note that an identical routine with name FPOLY1 must be compiled if using
 the optimum phase overlay (i.e. CALL FPOLY1 from routine FINISH).  However,
 most systems can fit the whole program into core without 
 overlay (size of COMMON dominates if N $>$ 1000).
\bigskip
\bigskip
\centerline {CHOICE OF INPUT PARAMETERS}
\bigskip

        The integration is controlled by several parameters
    specified as input.  Among these ETAI, ETAR, DTMIN and DELTAN are
  the most important ones, whereas SMIN, RMIN, CMSEP2, ECLOSE, DFMIN and DFMAX
 play a secondary role.  Irregular and regular time-steps are obtained 
from a relative convergence criterion of the type
STEP = (ETA$\ast$(F$\ast$F2DOT + FDOT$\ast$$\ast$2)/(FDOT$\ast$F3DOT +
  F2DOT$\ast$$\ast$2))$\ast$$\ast$0.5, based on force derivatives 
 (converted from appropriate differences).  Regularized
 time-steps (non-physical units) are normally computed from the expression
  DTAU = 1.0/(DELTAN$\ast$SQRT(2.0$\ast$$\vert$H$\vert$), where H is
 the binding energy per unit mass.  This is reduced by the factor
    1.0/(1.0 + 1000.0$\ast$GAMMA)$\ast$$\ast$0.33, where GAMMA is the relative two-body
 perturbation.  One weakly perturbed binary orbit requires 6.28$\ast$DELTAN steps.  A
  procedure for unperturbed motion of hard binaries is implemented
 when GAMMA $<$ DFMIN at the apocentre.  Note that without stabilization
 (option 14) numerical errors may exceed the dynamical effects.

   Suggestions for the dimensionless parameters: ETAI = 0.03, ETAR = 0.08,
   DELTAN = 8.0, CMSEP2 = 70.0, DFMIN = 1.0E-06, DFMAX = 0.01,
   QE = 1.0E--04 (per TCR).  For increased accuracy try ETAI = 0.02,
 ETAR = 0.06, DELTAN = 10.0, CMSEP2 = 100.0, QE = 3.0E--05.  Option 4 is used to skip
 full predictor loops for neighbour numbers $>$ KZ(4).  If KZ(4) $>$ 0 it is
 set to the average value ($<$NNB$>$) every output.  About 3/4 of the regular 
 steps can be skipped, unless a massive body is present (check counter 
 NSTEPN(12)).  Note that the calculation is halted for relative (kinetic) energy
 errors DE/KE $>$ 5.0$\ast$QE unless option 2 is non-zero (restart facility).

  As time-step criterion for new regularizations take
 DTMIN = 0.02$\ast$(RMIN$\ast$$\ast$3/ $<$M$>$)$\ast$$\ast$0.5, where $<$M$>$  
is the mean mass.  A good
 choice for the regularizing separation is RMIN = 4$\ast$$<$R$>$/N, corresponding
 to the close encounter definition ($<$R$>$ = half-mass radius).  Alternatively,
 try RMIN = $<$R$>$/N for small N or high central densities.  SMIN
 is used for retaining neighbours with small steps and also for
 omitting large third-order force derivative corrections which might
 otherwise produce small regular steps.  Try using
 SMIN = 2$\ast$DTMIN.  The
 maximum step $(R_0/M)^{1/2}$/DELTAN (changed from BETA June 1988)
 used for soft binaries {\it \&} weak hyperbolic encounters is
 modified by $(R_0/R)^{1/2}$ if $R > R_0$.  If
 option KZ(17) $>$ 0, ETAI {\it\&} ETAR are increased or 
 decreased in routine MAINPT using the tolerance QE.  If option KZ(17) $>$ 1,  
 new values of RMIN {\it\&} DTMIN 
 include the cube root of the central density contrast (small N
 limit RMIN $<$ 0.01$\ast$$<$R$>$ adopted for increased efficiency).

 The regularization of soft binaries is terminated by the conditions
 GAMMA $>$ DFMAX {\it\&} R $>$ R0.  For hyperbolic encounters, the  
 condition R $>$ RMIN is used.  The regularization
 of a hard binary (convention R0 = RMIN) is terminated if 
 GAMMA $>$ 0.5 and the perturber forms a dominant pair with
 either component.  From the definition of a hard binary and assuming 
 equal masses, take ECLOSE =
 N$\ast$$<$M$>$/2$\ast$$<$R$>$ for cluster calculations.  In
 order to have about NNB neighbours for constant density, take an initial
 neighbour radius RS0 = (NNB/N)$\ast$$\ast$0.33$\ast$$<$R$>$.  
 Choose a maximum neighbour number of at least
 NNBMAX = 10 + N$\ast$$\ast$0.5.  The present version with
 LIST(40,1010) is designed for NNBMAX $<$ 38 (see the special procedure at label 
 79 of FINISH).  NNBMAX = 37 permits correct treatment of at least two
 regularization terminations among the neighbours of one body during a regular
 time-step.  Note that one extra neighbour location is required each time the
 c.m. of a regularized pair is replaced by the two components.

 The parameters RBAR (set in TIDAL(4)) {\it\&} ZMBAR may be used to 
 include a galactic tidal
 field (see 1985 book article).  For open clusters with N = 1000 members
 a scale factor RBAR = 2 -- 3 (pc) is a good
 choice.  The mean stellar mass is usually taken as ZMBAR = 0.5 (solar
 masses).  A consistent tidal radius is calculated from these parameters.  For
 isolated systems take RBAR = 0 {\it\&} ZMBAR $>$ 0, in which case 
 routine MAINPT sets RTIDE = 10$\ast$$<$R$>$.
 Note that 2$\ast$RTIDE is used as an escape criterion, together 
 with a positive energy condition (isolated systems only).

 One input card is also included for interstellar clouds (option 13).  Runs
 have been made with the following parameters: NCL = 5, RB2 = 28, VCL = 10,
 SIGMA = 0 (or VCL = 4 {\it\&} SIGMA = 6), CLM = 50 -- 500 {\it\&} RCL2 = 4 -- 8.  
 Use standard integration parameters and a large energy tolerance 
 QE since there is no integral of motion.
\bigskip
\bigskip
\centerline {TRIPLE}
\bigskip

 This is a complete package for treating three-body collisions to be included with program NBODY5                
   and is described in the 1985 article.  It is based on the Aarseth -- Zare (AZ) regularization method                          
   ({\it Celestial Mechanics} 10, 185) and consists of the following subroutines:  TRIPLE, START3, TRANS3,                    
   QDERIV and DIFSY1.  The triple regularization is only implemented if option KZ(5) $>$ 0.  Suitable                     
   critical encounters are most readily identified by small c.m. steps, using the condition STEP(I) $<$                   
   DTMIN when a binary is at apocentre.  If satisfied, a search is then made for the intruding third                    
   body as well as the dominant perturber of the triple motion.  The critical configuration is accepted                 
   for special treatment if the impact parameter is less than the semi-major axis and the nearest                       
   perturber effect is suitably small.  Before transferring control to the triple package the binary                    
   regularization is terminated by a call to routine FINISH from the main routine.                                      

 Routine TRIPLE controls the decision-making and lists the COMMON variables.  First                      
   it calls routine START3 which sets the
   initial conditions in the local centre of mass frame and calculates the maximum size for a nearest                   
   neighbour perturbation of 1.0E--04.  Routine TRANS3 then transforms the local coordinates and                         
   velocities to regularized variables.  An absolute tolerance of 1.0E-10 is specified for the                          
   Bulirsch--Stoer integrator DIFSY1.  The resulting relative energy error is usually smaller than the                   
   tolerance but even so the high-order integrator is quite fast and not much time can be saved by                      
   reducing the accuracy.  The present formulation integrates 17 equations as described in the AZ paper                 
   but uses a modified time transformation given by TPR = R1$\ast$R2/(R1 + R2)$\ast$$\ast$0.5 which permits                            
   stabilization (see the 1976 Cortina paper for details).  The decision-making uses the three mutual                   
   separations R1, R2, R3, where the latter corresponds to the distance between bodies M(1) {\it\&} M(2).  A                  
   switch to a new reference body M(3) occurs if R3 becomes the smallest distance.  There are three                     
   termination criteria:  R3 $>$ RMAX, R1 + R2 + R3 $>$ 3$\ast$RSTAR and one body escaping, or time interval $>$                   
   100$\ast$TCR.  Here RMAX is the largest permitted size for an unperturbed system (governed by the dominant                
   perturber), RSTAR is the gravitational radius (defined by the sum of the mass products over the total                
   energy) and TCR is the local crossing time (different from                                                                          
   TCR of COMMON5).

 After transformation to regularized variables by routine TRANS3 the integration is carried out by                 
   routine DIFSY1.  The latter calls routine QDERIV to evaluate the right-hand sides of the first-order                 
   differential equations at chosen sub-intervals.  On each return to routine TRIPLE the termination                    
   distance criterion is first checked.  If instead R1 + R2 + R3 $>$ 3$\ast$RSTAR the escape condition is                      
   evaluated after transformation to physical variables.                                                                

 The three-body regularization is normally terminated by the escape of one body from the triple.  From                     
   the rigorous criterion of Standish ({\it Celestial Mechanics} 4, 44) escape occurs if the velocity of                 
   the third body exceeds a critical value given by the masses and the distance to the binary (evaluated                
   outside RSTAR for RIDOT $>$ 0).  The integration is extended for a short duration (up to                
   10 tries) if the binary components are inside half the semi-major axis at termination in order to                    
   get a more reliable value of the new binding energy.  One line of diagnostic information is                       
   printed at the end:  BINARY (text), component names, binary mass, semi-major axis, eccentricity,                     
   binding energy (in terms of the triple), relative perturbation on binary, separation of the binary                   
   components, mass of the third body, its distance to the binary and the relative energy of the triple                 
   system (scaled by the total energy).  If option KZ(5) $>$ 1 one further line is printed giving details                 
   of the integration itself, such as the number of switching transformations and minimum separations.  This                  
   condition is also used to give a line of output at the beginning of regularization (printed in                  
   routine INTGRT) as follows:  BEGIN TRIPLE (text), IPAIR (binary index), TIME, H(IPAIR), R(IPAIR),                    
   mass of binary and third body, external perturbation of dominant neighbour, third body separation and                
   impact parameter, relative energy of third body (scaled by binary energy) and number of active binary                
   perturbers.                                                                                                          

 Following termination, a second call to routine START3 transforms the local variables back to                   
  corresponding global values, taking into account possible switching of
   the particles.  This routine also sets new force polynomials (F {\it\&} FDOT only, higher orders = 0) and                  
   conservative time-steps for the more distant third body.  The binary itself is accepted for KS                       
   regularization by routine TRANSF immediately on return to the main program, whereupon the standard                   
   decision-making takes over.                                                                                          
\bigskip
\bigskip
\centerline {TETRAD}
\bigskip

 This is a complete package for four-body collisions to be included with program NBODY5.  It uses                
   the Heggie global regularization method ({\it Celestial Mechanics} 10, 217) as formulated                  
by Mikkola ({\it M. N.}
   215, 171).  Apart from the decision-making and interfacing with the N-body method all programming is                 
   due to Mikkola.  For this reason most parts are without comments but the program is very                     
   efficient and can be used as a black box.  The relevant main COMMON
   variables are listed in routine TETRAD.  This program is intended for two colliding binaries                       
   treated as an isolated system.  The N-body procedures are similar to the case of three colliding                     
   particles, except here the intruder is also a binary.  Critical configurations are again identified                  
   in routine INTGRT for small c.m. steps (STEP(I) $<$ DTMIN), using the same procedure as for triple                     
   collisions.  Thus option KZ(5) has the same meaning in both cases.  Note that the critical impact                    
   parameter is now modified by the size of the second binary to give an                    
   increased cross-section (see 1985 article for details).  If
   option KZ(5) $>$ 1 routine INTGRT prints the same line as for triples, except it starts with the text                  
   BEGIN TETRAD.  Before transferring control to the tetrad package the regularization of both binaries                 
  is terminated in the main routine by two successive calls to routine FINISH.                                                    

 Routine TETRAD controls the main flow.  First it calls routine START4 which sets the initial                    
   conditions in the local centre of mass frame and evaluates the maximum size for a nearest neighbour                  
   perturbation of 1.0E--04 (as in TRIPLE).  An absolute tolerance of 1.0E--06 is adopted                          
   for the Bulirsch--Stoer
   integrator.  This is not as accurate as for triple collisions but should still be satisfactory.  Note                
   that there are now 49 first-order equations to be integrated, hence the choice of tolerance.  In the                 
   absence of a rigorous escape criterion for four bodies there are just two termination criteria:  DMAX                
   $>$ RMAX, and time interval $>$ 100$\ast$TCR.  Here DMAX is the largest                 
   distance between two particles and RMAX
   is the maximum permitted size for an unperturbed system or at most 3$\ast$RSTAR, where RSTAR is the                       
   generalized gravitational radius.  As before TCR is the local                                          
  crossing time (no COMMON5 conflict).

 The four-body regularization is normally terminated by the distance criterion.  If desirable, it                
   would be possible to implement an approximate escape test by considering the smallest binary as one                  
   body and then use the Standish criterion.  Again the integration is extended for a short interval (up                
   to 10 or 20 tries depending on the configuration) if any binary components are inside half the                       
   semi-major axis.  One line of output is printed if the relative energy increase of the most energetic                
   binary exceeds 10 percent.  Note that at least one binary normally survives the collision but it may                 
   consist of new components arising from exchange.  The diagnostic output contains the following                          
   information:  BINARY (text), binary mass, semi-major axis, eccentricity, binding energy (in terms of                 
   the tetrad) and perturbation on the closest pair, perturbation on the second binary (= 0 if none),                   
   separation of components 3 {\it\&} 4, their relative energy (scaled by the tetrad), binding energy of the                  
   relative motion (scaled by the initial binary), and finally the total energy of the tetrad system                     
   (scaled by the total energy).  If option KZ(5) $>$ 1 a further line is also printed giving the                         
   following details:  END TETRAD (text), indices of the final configuration (closest pair first),                      
   separations of closest pair (R12) as well as R13 {\it\&} R24, RSTAR (gravitational radius), time interval                  
   (scaled by local crossing time), number of DIFSY calls, relative energy change of the hardest binary                 
   and energy of the tetrad system (scaled by total energy).                                                            

 Upon termination, a second call to routine START4 transforms the local variables back to global                 
   values.  Two of the particles are now initialized for direct N-body integration, again omitting                      
   second and third order derivatives and taking conservative time-steps.  Note that these particles may                
   still constitute a binary which may be selected for regularization later.  The closest particle pair                 
   is accepted for KS regularization by calling routine TRANSF immediately on return to the main program                
   which continues in the standard way.                                                                                 
\bigskip
\bigskip
\centerline {MERGE}
\bigskip
 The third sub-program deals with stable hierarchical configurations and should also be included                 
   with program NBODY5.  The basic idea is to adopt the c.m. approximation for                     
  the inner binary of a stable triple,
   thereby enabling the outer component to be regularized instead.  This has been                        
   generalized to the case where the outer component is also a binary.  The                    
   search procedure in routine INTGRT is exactly the same as for triple and tetrad collisions.  Thus                    
   only compact configurations with STEP(I) $<$ DTMIN for the binary c.m. are selected.  The main                         
   criterion for accepting a stable triple is that the pericentre of the outer orbit, or impact                         
   parameter, should exceed the apocentre distance of the inner binary by a safety factor.  The adopted                 
   ratio of 5 is based on three-body experiments.  An allowance is made for the perturbation effect of                  
   the most dominant neighbour.  The hierarchical configuration is not accepted for an extended outer                   
   orbit (i.e., soft binary)
   since sufficient perturbers may not be available in the c.m. neighbour list.  If option KZ(5) $>$                
   1 routine INTGRT prints the same line as before, except it starts with the text BEGIN MERGE.                         

 Routine MERGE is called from the main program with COMMON variables ICOMP and JCOMP holding the                 
   binary pair index (IPAIR) and index of the outer component.                     
  If JCOMP $>$ N the outer binary is specified as unperturbed with
  large STEP. The inner binary is then terminated in the standard
  way by calling routine FINISH.
    The original binary configuration is saved in a new                     
   labelled COMMON (also added in routine MYDUMP).  A ghost of zero mass and large coordinates is                       
   created for index JCOMP to have an available location when the merged binary is activated again.  The                
   integration variables are set to zero, with large T0(JCOMP),
   and the ghost is removed from all neighbour lists.  The new                     
   binary is regularized by a call to routine TRANSF.  Its c.m. name is assigned a negative                  
   value which enables the corresponding ghost particle to be identified.  The minimum impact parameter                 
   (5$\ast$A$\ast$(1 -- E)) used for termination is set in variable R0(NPAIRS).  Together with the negative name                   
   convention this avoids adding new variables to the N-body COMMON.  Finally the total energy is                       
   updated by subtracting the old binary energy as well as the tidal energy correction due to the c.m.                  
  approximation.  Hence energy conservation can still be used.

 Routine INTGRT contains an appropriate termination test for hierarchical configurations.  An                    
   extra check is included at each binary apocentre if the corresponding c.m. name is negative.  It                     
   consists of comparing the predicted pericentre (2$\ast$SEMI -- R(IPAIR)) with R0(IPAIR)$\ast$(1 +                             
   GAMMA(IPAIR)) which allows for perturbation effects.  On termination and return to the main program                  
   control is transferred to routine RESET instead of routine FINISH.                                                   

 Routine RESET restores the old hierarchical configuration of a regularized pair with index IPAIR                
   held in variable ICOMP.  Up to 10 cases are allowed.  First the current location in the merger table                 
   is determined.  The perturbers are saved for potential energy correction, reducing any index J $>$ N +                 
   IPAIR.  A call to the N-body routine FINISH terminates the outer pair.  The correct location of the                  
   ghost particle is found by testing for zero mass and appropriate name.  This component (denoted                      
   JCOMP) now becomes the second regularization component, whereas ICOMP is the first and the old JCOMP                 
   in 2$\ast$NPAIRS + 2 is the outer component with new force polynomials.  The old masses are introduced for                
   ICOMP {\it\&} JCOMP, whereas coordinates and velocities are given by the current c.m. values combined with                 
   the relative quantities preserved in COMMON/BINARY/.  Index JCOMP is added to all neighbour lists                    
   containing ICOMP (combined into one c.m. by TRANSF).  The old binary is regularized by calling                   
  routine TRANSF.  An unperturbed restart is prepared if the outer
  component is a c.m. particle, and appropriate potential energy
  corrections are made.
   The total energy is modified by the new binding energy and the net potential                 
   energy change
   due to resolving the inner pair.  If option KZ(5) $>$ 1 a line of output is printed with the                    
   following diagnostic:  END MERGER (text), merger index, TIME, component masses, separation of outer                  
   component and its semi-major axis, binding energy ratios of inner binary and outer component (scaled                 
   by total energy and inner binary, respectively), relative perturbation of the new inner and old outer                
   binary, and finally the number of old perturbers.                                                                    
 Likewise, information about the second merged binary is also
 printed at termination as follows: SECOND MERGED BINARY (text),
 pair index, binding energy per unit mass, component masses, binding
 energy ratio of inner binary and second binary, binding energy of
 second binary (scaled by total energy), pair separation, and
 relative perturbation.

 Routine RESET ends by reducing the merger counter NMERGE.  All relevant tables in COMMON/BINARY/                
   are also updated if JMERGE $<$ NMERGE + 1.  The index JMERGE holds the old number of merged pairs with                 
   negative c.m. names.  Thus merged pairs may be removed by routine ESCAPE, using a special                        
   procedure (distant ghosts with zero mass are otherwise retained).  At the end control                    
   passes back to the main program which then calls routine INTGRT.                                                                             
\bye
