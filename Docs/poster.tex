
\documentclass[12pt]{article}
\usepackage{color}
\usepackage{graphicx}
\pagestyle{plain}
\pagestyle{empty}
\textwidth 14cm
\textheight 23cm
\voffset=-1.0truecm
\hoffset=-1truecm
\definecolor{yuck}{rgb}{0.63,0.125,0.94}
\definecolor{yuck2}{rgb}{0.80,0.125,0.565}
\def\ZZ#1{$\scriptstyle #1$}

\begin{document}

\centerline {\LARGE \color{red} NBODY6}
\bigskip
\medskip
\centerline {\LARGE \color{blue} A new star cluster simulation code}
\bigskip
\bigskip
\centerline {\Large \color{green} Sverre J Aarseth}
\bigskip
\centerline {\Large \color{black}Institute of Astronomy, Cambridge, UK}
\bigskip
\bigskip
{\large
\noindent
{\bf Abstract}.~~A new direct $N$-body code, \ZZ {NBODY6},
supersedes the similar code \ZZ {NBODY5} which
has been in general use since the early 1980s.
This code incorporates several recent developments which result in
improved accuracy as well as performance.
It is suitable for studying a variety of problems involving point-mass
interactions, with emphasis on realistic star cluster models.

}
\bigskip
{\Large
The new code \ZZ {NBODY6} is constructed along similar lines to
\ZZ {NBODY5} but employs a number of improved procedures.
It consists of some 28800 lines of Fortran 77 divided into 211 routines
and contains the following main features:

\color{red}
\medskip
{$\bullet$}~Hermite integration with hierarchical block-steps

\medskip
{$\bullet$}~Ahmad-Cohen neighbour scheme

\medskip
{$\bullet$}~External tidal field -- smooth and irregular component

\color{blue}
\medskip
{$\bullet$}~Stumpff KS formulation with Hermite integrator

\color{green}
\medskip
{$\bullet$}~Stability of hierarchical systems

\color{cyan}
\medskip
{$\bullet$}~Chain regularization of compact subsystems

\color{magenta}
\medskip
{$\bullet$}~Slow-down by adiabatic invariance for KS and chain

\color{yuck}
\medskip
{$\bullet$}~Synthetic stellar evolution with arbitrary metallicity

\color{yuck2}
\medskip
{$\bullet$}~Instantaneous tidal circularization and collisions

\bigskip

\color{red}
Following several years experience with \ZZ {NBODY4} on the HARP
special-purpose computer (Aarseth 1996), the fourth-order force
polynomial with divided differences used by \ZZ {NBODY5} has been 
replaced by Hermite integration of the same order (Makino 1991).
Like its predecessor, the code is based on the Ahmad-Cohen (1973)
neighbour scheme which speeds up the force calculation and
enables larger systems to be investigated on conventional machines.
External perturbations due to a smooth tidal field or irregular
interstellar clouds are included for circular cluster orbits.

\color{blue}
\medskip
Close two-body encounters and hard binaries are still studied by the
classical KS regularization method. However, we now employ a formulation
with the solution expanded in Taylor series which is modified by Stumpff
functions and combined with a Hermite iterative development for the
corrector (Mikkola \& Aarseth 1998).
This treatment yields machine accuracy in the limit of small
perturbations at no extra computational cost since larger steps are
permitted and the perturbing force is only evaluated once even for large
perturbations.

\color{green}
\medskip
The formation of long-lived hierarchical triples by binary-binary encounters
poses severe time-scale problems for long-term integration.
Provided the outer pericentre distance is sufficiently large, the inner
semi-major axis is not subject to secular effects.
A new semi-analytical stability criterion (Mardling \& Aarseth 1999) is
implemented to select systems where the inner binary can be replaced
by the centre-of-mass approximation, leaving the outer orbit as the
second member of a KS solution until the condition is no longer satisfied
(e.g. increased eccentricity or large perturbation).
Likewise, systems of higher multiplicity, such as [B-B] or [[B-B]-B], are
considered in a similar manner.

\color{cyan}
\medskip
Strong interactions between hard binaries and single stars or other
binaries are treated by chain regularization (Mikkola \& Aarseth 1993).
Only one compact subsystem with external perturbers is considered  at
a time but typical durations are short.
In addition, one unperturbed triple and one quadruple system can also be
studied simultaneously by multiple regularization techniques.
Such procedures offer two significant advantages over direct calculations:
(i) notwithstanding chaos, extremely energetic interactions can be studied
with confidence, and (ii) the use of well-defined variables is highly
beneficial for carrying out dissipative modifications of a singular nature.

\color{magenta}
\medskip
The principle of adiabatic invariance has been introduced to slow down the
motion for small perturbations (Mikkola \& Aarseth 1996), both in the case
of KS and chain regularization.
By scaling the perturbing force we neglect short-term
fluctuations of the two-body elements while preserving secular effects.
This speeds up the integration of hard binaries and facilitates the
treatment of subsystems containing arbitrarily small periods.

\color{yuck}
\medskip
Realistic simulations of star clusters require consideration of stellar
evolution effects, since the masses and radii may change significantly
during the cluster life-time.
The earlier scheme of synthetic stellar evolution for solar composition
(Tout et al. 1997) has been replaced by a treatment employing an arbitrary
(fixed) value of the metallicity (Hurley et al. 1999) which enables
globular cluster stars to be represented.
Both fast mass loss due to supernovae explosions and slow wind loss
from giants are included.

\color{yuck2}
\medskip
An optional procedure for instant tidal circularization permits the
subsequent stage of gravitational or magnetic braking to be considered.
Several types of stellar collisions are also catered for, although the
relevant cross-sections are awaiting improvements.
All the procedures involving mass loss or dissipation include appropriate
corrections in order to maintain formal energy conservation.

\color{blue}
\medskip
In conclusion, the code is both more accurate and slightly
faster than \ZZ {NBODY5} on a standard workstation or laptop, as well as
being more versatile.
It is available on request by public ftp, together with some basic
documentation.

}
\bigskip

\color{black}
\begin{thebibliography}{}
\addtolength{\leftmargin}{0.7cm}
\setlength{\itemindent}{-0.7cm}

{\Large
\item[] Aarseth, S.J., 1996, in Dynamical Evolution of Star Clusters, eds
P. Hut \& J. Makino (Kluwer), p. 161
\item[] Ahmad, A. \& Cohen, L., 1973, {\it J. Comp. Phys.}, {\bf 12}, 389
\item[] Hurley, J.R., Tout, C.A. \& Pols, O., 1999, (unpublished)
\item[] Makino, J., 1991, {\it Astrophys. J.}, {\bf 369}, 200
\item[] Mardling, R.A. \& Aarseth, S.J., 1999, (in preparation)
\item[] Mikkola, S. \& Aarseth, S.J., 1993, {\it Cel. Mech.}, {\bf 57}, 439
\item[] Mikkola, S. \& Aarseth, S.J., 1996, {\it Cel. Mech.}, {\bf 64}, 197
\item[] Mikkola, S. \& Aarseth, S.J., 1998, {\it New Astron.}, {\bf 3}, 309
\item[] Tout, C.A., Aarseth, S.J., Pols, O.R. \& Eggleton, P.P., 1997,
{\it Mon. Not. R. Astr. Soc.}, {\bf 291}, 732

}
\end{thebibliography}
\end{document}
